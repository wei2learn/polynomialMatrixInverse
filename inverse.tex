%% LyX 2.0.5.1 created this file.  For more info, see http://www.lyx.org/.
%% Do not edit unless you really know what you are doing.
\documentclass[english]{elsarticle}
\usepackage[T1]{fontenc}
\usepackage[latin9]{inputenc}
\usepackage{color}
\usepackage{verbatim}
\usepackage{prettyref}
\usepackage{float}
\usepackage{amsthm}
\usepackage{amsmath}
\usepackage{amssymb}

\makeatletter

%%%%%%%%%%%%%%%%%%%%%%%%%%%%%% LyX specific LaTeX commands.
\floatstyle{ruled}
\newfloat{algorithm}{tbp}{loa}
\providecommand{\algorithmname}{Algorithm}
\floatname{algorithm}{\protect\algorithmname}

%%%%%%%%%%%%%%%%%%%%%%%%%%%%%% Textclass specific LaTeX commands.
\floatstyle{ruled}
\newfloat{algorithm}{tbp}{loa}
\floatname{algorithm}{Algorithm}
\usepackage{algorithmic}
\newcommand{\forbody}[1]{ #1 \ENDFOR }
\newcommand{\ifbody}[1]{ #1  \ENDIF}
\newcommand{\whilebody}[1]{ #1  \ENDWHILE}
\renewcommand{\algorithmicprint}{\textbf{draw}}
\renewcommand{\algorithmicrequire}{\textbf{Input:}}
\renewcommand{\algorithmicensure}{\textbf{Output:}}

\theoremstyle{plain}
\ifx\thechapter\undefined
\newtheorem{thm}{\protect\theoremname}
\else
\newtheorem{thm}{\protect\theoremname}[chapter]
\fi
  \theoremstyle{plain}
  \newtheorem{lem}[thm]{\protect\lemmaname}
  \theoremstyle{definition}
  \newtheorem{defn}[thm]{\protect\definitionname}

%%%%%%%%%%%%%%%%%%%%%%%%%%%%%% User specified LaTeX commands.
%\usepackage{yjsco}\journal{JournalofSymbolicComputation}

\renewcommand{\algorithmicrequire}{\textbf{Input:}}\renewcommand{\algorithmicensure}{\textbf{Output:}}
%\renewcommand{\algorithmicensure}{\textbf{if}}\renewcommand{\algorithmicensure}{\textbf{Uses:}}

%\def\diag{\mbox{diag}}\def\cdeg{\qopname\relax n{cdeg}}
\def\MM{\qopname\relax n{MM}}\def\M{\qopname\relax n{M}}
%\def\ord{\qopname\relax n{ord}}

\def\StorjohannTransform{\qopname\relax n{StorjohannTransform}}\def\TransformUnbalanced{\qopname\relax n{TransformUnbalanced}}\def\rowDimension{\qopname\relax n{rowDimension}}\def\columnDimension{\qopname\relax n{columnDimension}}\DeclareMathOperator{\re}{rem}\DeclareMathOperator{\coeff}{coeff}\DeclareMathOperator{\lcoeff}{lcoeff}\def\mab{\qopname\relax n{orderBasis}}\def\mmab{\qopname\relax n{FastBasis}}\def\umab{\qopname\relax n{UnbalancedFastBasis}}\newcommand{\bb}{\\}
\def\mnb{\qopname\relax n{MinimaKernelBasis ~ }}
\DeclareMathOperator{\mnbr}{MinimaKernelBasisReversed}
%\def\rdeg{\qopname\relax n{rdeg}}
\DeclareMathOperator{\colBasis}{colBasis}
\DeclareMathOperator{\lcm}{lcm}
\DeclareMathOperator{\inv}{inverse}



\newcommand{\arne}[1]{{\color{blue}\it {\bf Arne:} #1 }}
\newcommand{\wei}[1]{{\color{red}\it {\bf Wei:} #1}}
\newcommand{\george}[1]{{\color{green}\it {\bf George:} #1}}
\def\newblock{\hskip .11em plus .33em minus .07em}


\newcommand{\Z}{\mathbb{Z}}
\newcommand{\Q}{\mathbb{Q}}
\newcommand{\N}{\mathbb{N}}
\newcommand{\K}{ {\rm K}}
\newcommand{\revCol}{ {\rm revCol}}
\newcommand{\bigO}{\mathcal{O}}
\newcommand{\tbigO}{\widetilde{\mathcal{O}}}
\newcommand{\cL}{\mathcal{L}}
\newcommand{\ocL}{\overline{\mathcal{L}}}
\newcommand{\tcL}{\widetilde{\mathcal{L}}}
\newcommand{\cA}{\mathcal{A}}
\newcommand{\oA}{\overline{A}}
\newcommand{\GL}{{\rm GL}\,}
\newcommand{\rank}{{\rm rank}\,}
\newcommand{\cdeg}{{\rm cdeg}\,}
\newcommand{\rdeg}{{\rm rdeg}\,}
\newcommand{\diag}{{\rm diag}\,}
\newcommand{\val}{{\rm val}\,}
\newcommand{\ord}{{\rm ord}\,}
\newcommand{\abs}[1]{\lvert#1\rvert}


\newrefformat{eq}{\textup{(\ref{#1})}}
\newrefformat{lem}{Lemma \ref{#1}}
\newrefformat{cla}{Claim \ref{#1}}
\newrefformat{thm}{Theorem \ref{#1}}
\newrefformat{cha}{Chapter \ref{#1}}
\newrefformat{sec}{Section \ref{#1}}
\newrefformat{rem}{Remark \ref{#1}}
\newrefformat{fac}{Fact \ref{#1}}
\newrefformat{sub}{Subsection \ref{#1}}
\newrefformat{cor}{Corollary \ref{#1}}
\newrefformat{cond}{Condition \ref{#1}}
\newrefformat{con}{Conjecture \ref{#1}}
\newrefformat{def}{Definition \ref{#1}}
\newrefformat{pro}{Proposition \ref{#1}}
\newrefformat{alg}{Algorithm \ref{#1}}
\newrefformat{exm}{Example \ref{#1}}
\newrefformat{line}{line \ref{#1}}
\newrefformat{tab}{Table \ref{#1} on page \pageref{#1}}
\newrefformat{fig}{Figure \ref{#1} on page \pageref{#1}}

\makeatother

\usepackage{babel}
  \providecommand{\definitionname}{Definition}
  \providecommand{\lemmaname}{Lemma}
\providecommand{\theoremname}{Theorem}

\begin{document}

\title{An efficient, deterministic algorithm for inversion of a matrix polynomial}


\author{Wei Zhou, George Labahn, Arne Storjohann}


\address{Cheriton School of Computer Science, University of Waterloo, Waterloo
ON, Canada N2L 3G1}


\ead{\{w2zhou,glabahn,astorjoh\}@uwaterloo.ca}
\begin{abstract}
In this paper we present a deterministic algorithm for the computation
of the inverse of an $n\times n$ matrix of polynomials over a field.
The algorithm is deterministic with complexity $O^{\sim}\left(n^{3}s\right)$
field operations. Here $s$ is the minimum of the average of the column
degrees and the average of the row degrees of the input matrix and
$O^{\sim}$ is just $O$ with log factors omitted. In addition, our
algorithm can also be used to find the largest invariant factor of
a polynomial matrix, again deterministically, with a cost of $O^{\sim}\left(n^{\omega}s\right)$,
where $\omega$ is the exponent of matrix multiplication. %
\thanks{Do I have the correct complexity here?%
} 
\end{abstract}
\maketitle

\section{Introduction}

Let $\mathbf{F}\in\mathbb{K}\left[x\right]^{n\times n}$ be an $n\times n$
matrix of polynomials over an abstract field $\mathbb{K}$. We consider
the problem of computing the inverse of $\mathbf{F}$. Jeannerod
and Villard \citet{jeannerod-villard:05} give a deterministic algorithm
that can efficiently compute the inverse of $\mathbf{F}$ with a cost
of $O^{\sim}\left(n^{3}d\right)$ field operations, provided that
the dimension of $\mathbf{F}$ is a power of $2$, and the intermediate
results have small and balanced degrees. Here $O^{\sim}$ is just
$O$ with log factors omitted. Previous to this the fastest deterministic
algorithms for matrix polynomial inversion had complexity $O^{\sim}\left(n^{\omega+1}d\right)$
\citep{burgisser,storjohann:phd2000}.

In this paper we extend the algorithm of Jeannerod and Villard to
one which is still deterministic and has complexity $O^{\sim}\left(n^{3}s\right)$
for arbitrary input matrix polynomials. Here $s$ is the average column
degree or the average row degree. We assume without loss of generality
that the column degrees has the smaller average and just work with
column degrees in this paper. Note that in the special case when all
the rows and columns of $\mathbf{F}$ have uniform degree $d$ then
the complexity of inversion becomes $O^{\sim}\left(n^{3}d\right)$.

Our algorithm makes use of a similar reduction used by Jeannerod and
Villard \citet{jeannerod-villard:05} which uses a sequence of kernel
basis computations to reduce inversion to the inversion of a diagonal
matrix. Our extension makes use of {\em shifted} kernel nullspace
bases combined with an alternate method for measuring size during
the reduction. This allows us to replace the kernel basis computation
and the matrix multiplications with the new algorithms from \citep{ZLS2012}.
In addition, whereas the algorithm of Jeannerod and Villard returns
two matrices $\mathbf{A},\mathbf{B}$ satisfying $\mathbf{F}^{-1}=\mathbf{A}\mathbf{B}^{-1}$,
our algorithm instead returns a list of matrices $\mathbf{A}_{1},\dots,\mathbf{A}_{\left\lceil \log n\right\rceil },\mathbf{B}$
satisfying $\mathbf{F}^{-1}=\mathbf{A}_{1}\cdots\mathbf{A}_{\left\lceil \log n\right\rceil }\mathbf{B}^{-1}$.
We can then compute the product $\mathbf{A}=\mathbf{A}_{1}\cdots\mathbf{A}_{\left\lceil \log n\right\rceil }$
with a cost of $O^{\sim}\left(n^{3}s\right)$. It is interesting to
note that the output $\mathbf{A}_{1},\dots,\mathbf{A}_{\left\lceil \log n\right\rceil },\mathbf{B}$
takes only $O(n^{2}s\log n)$ space, but the product $\mathbf{A}=\mathbf{A}_{1}\cdots\mathbf{A}_{\left\lceil \log n\right\rceil }$
can take $n^{3}s$ space.

As an added contribution we show that the reduction also allows us
to determine the largest invariant factor of a matrix of polynomials.
This largest invariant factor is the smallest polynomial $s$ such
that $s\cdot\mathbf{F}^{-1}$ has only polynomial entries. When viewed
in terms of the determinant the largest invariant factor plays the
same role as the minimal polynomial of a scalar matrix polynomial
when compared to the characteristic polynomial.%
\footnote{ This sentence leaves a bit to be desired. Improvements welcome. %
} In fact, when ${\mathbf{A}}=xI-V$ for some $V\in\mathbb{K}^{n\times n}$
then the largest invariant factor of ${\mathbf{A}}$ is the same as
the minimal polynomial of $V$.

The remainder of this paper is as follows. The inversion algorithm
is included in the next section followed in Section \ref{sec:complexity}
by the theorems giving the new complexity results. Section \ref{sec:invariants}
then gives our contribution with respect to the largest invariant
factor. The paper ends with a conclusion and topics for future research.


\section{Preliminaries}

In this paper computational cost is analyzed by bounding the number
of arithmetic operations in the coefficient field $\mathbb{K}$ on
an algebraic random access machine. We assume the cost of multiplying
two polynomial matrices with dimension $n$ and degree $d$ is $O^{\sim}(n^{\omega}d)$
field operations, where the multiplication exponent $\omega$ is assumed
to satisfy $2<\omega\le3$. We refer to the book by \citet{vonzurgathen}
for more details and references about the cost of polynomial multiplication
and matrix multiplication.

In this section we first describe the notations used in this paper,
and then give the basic definitions of {\em shifted degree} and
{\em kernel basis}, which are essential for our algorithm.


\subsection{Notations}

For convenience we adopt the following notations in this paper. 
\begin{description}
\item [{Comparing~Unordered~Lists}] For two lists $\vec{a}\in\mathbb{Z}^{n}$
and $\vec{b}\in\mathbb{Z}^{n}$, let $\bar{a}=\left[\bar{a}_{1},\dots,\bar{a}_{n}\right]\in\mathbb{Z}^{n}$
and $\bar{b}=\left[\bar{b}_{1},\dots,\bar{b}_{n}\right]\in\mathbb{Z}^{n}$
be the lists consists of the entries of $\vec{a}$ and $\vec{b}$
but sorted in increasing order. 
\[
\begin{cases}
\vec{a}\ge\vec{b} & \mbox{if }\bar{a}_{i}\ge\bar{b}_{i}\mbox{ for all }i\in\left[1,\dots n\right]\\
\vec{a}\le\vec{b} & \mbox{if }\bar{a}_{i}\le\bar{b}_{i}\mbox{ for all }i\in\left[1,\dots n\right]\\
\vec{a}>\vec{b} & \mbox{if }\vec{a}\ge\vec{b}\mbox{ and }\bar{a}_{j}>\bar{b}_{j}\mbox{ for at least one }j\in\left[1,\dots n\right]\\
\vec{a}<\vec{b} & \mbox{if }\vec{a}\le\vec{b}\mbox{ and }\bar{a}_{j}<\bar{b}_{j}\mbox{ for at least one }j\in\left[1,\dots n\right].
\end{cases}
\]

\item [{Summation~Notation}] For a list $\vec{a}=\left[a_{1},\dots,a_{n}\right]\in\mathbb{Z}^{n}$,
we write $\sum\vec{a}$ without index to denote the summation of all
entries in $\vec{a}$. %\item [{}]~

\item [{Compare~a~List~with~a~Integer}] For %a list 
$\vec{a}=\left[a_{1},\dots,a_{n}\right]\in\mathbb{Z}^{n}$ and $c\in\mathbb{Z}$,
we write $\vec{a}<c$ to denote $\vec{a}<\left[c,\dots,c\right]$,
and similarly for $>,\le,\ge,=$. 
\end{description}

\subsection{Shifted Degrees}

Our methods depend extensively on the concept of {\em shifted}
degrees of polynomial matrices \citet{BLV:1999}. For a column vector
$\mathbf{p}=\left[p_{1},\dots,p_{n}\right]^{T}$ of univariate polynomials
over a field $\mathbb{K}$, its column degree, denoted by $\cdeg\mathbf{p}$,
is the maximum of the degrees of the entries of $\mathbf{p}$, that
is, 
\[
\cdeg~\mathbf{p}=\max_{1\le i\le n}\deg p_{i}.
\]
The \emph{shifted column degree} generalizes this standard column
degree by taking the maximum after shifting the degrees by a given
integer vector that is known as a \emph{shift}. More specifically,
the shifted column degree of $\mathbf{p}$ with respect to a shift
$\vec{s}=\left[s_{1},\dots,s_{n}\right]\in\mathbb{Z}^{n}$, or the
\emph{$\vec{s}$-column degree} of $\mathbf{p}$ is 
\[
\cdeg_{\vec{s}}~\mathbf{p}=\max_{1\le i\le n}[\deg p_{i}+s_{i}]=\deg(x^{\vec{s}}\cdot\mathbf{p}),
\]
where 
\[
x^{\vec{s}}=\diag\left(x^{s_{1}},x^{s_{2}},\dots,x^{s_{n}}\right)~.
\]
For a matrix $\mathbf{P}$, we use $\cdeg\mathbf{P}$ and $\cdeg_{\vec{s}}\mathbf{P}$
to denote respectively the list of its column degrees and the list
of its shifted $\vec{s}$-column degrees. When $\vec{s}=\left[0,\dots,0\right]$,
the shifted column degree specializes to the standard column degree.
The shifted row degree of a row vector \textbf{$\mathbf{q}=\left[q_{1},\dots,q_{n}\right]$}
is defined similarly as 
\[
\rdeg_{\vec{s}}\mathbf{q}=\max_{1\le i\le n}[\deg q_{i}+s_{i}]=\deg(\mathbf{q}\cdot x^{\vec{s}}).
\]


Shifted degrees have been used previously in polynomial matrix computations
and in generalizations of some matrix normal forms \citet{BLV:jsc06}.
The shifted column degree is equivalent to the notion of \emph{defect}
commonly used in the literature.

The usefulness of the shifted degrees can be seen from their applications
in polynomial matrix computation problems \citep{ZL2012,ZLS2012}.
One of its uses is illustrated by the following lemma from \citet[Chapter 2]{zhou:phd2012},
which can be viewed as a stronger version of the predictable-degree
property \citet{kailath:1980}. 
\begin{lem}
\label{lem:predictableDegree}Let $\mathbf{A}\in\mathbb{K}\left[x\right]^{m\times n}$
be a $\vec{u}$-column reduced matrix with no zero columns and with
$\cdeg_{\vec{u}}\mathbf{A}=\vec{v}$. Then a matrix $\mathbf{B}\in\mathbb{K}\left[x\right]^{n\times k}$
has $\vec{v}$-column degrees $\cdeg_{\vec{v}}\mathbf{B}=\vec{w}$
if and only if $\cdeg_{\vec{u}}\left(\mathbf{A}\mathbf{B}\right)=\vec{w}$. 
\end{lem}
Another essential fact needed in our algorithm, also based on the
use of the shifted degrees, is the efficient multiplication of matrices
with unbalanced degrees \citep[Theorem 3.7]{zhou:phd2012}. 
\begin{thm}
\label{thm:multiplyUnbalancedMatrices} Let $\mathbf{A}\in\mathbb{K}\left[x\right]^{m\times n}$
with $m\le n$, $\vec{s}\in\mathbb{Z}^{n}$ a shift with entries bounding
the column degrees of $\mathbf{A}$ and $\xi$, a bound on the sum
of the entries of $\vec{s}$. Let $\mathbf{B}\in\mathbb{K}\left[x\right]^{n\times k}$
with $k\in O\left(m\right)$ and the sum $\theta$ of its $\vec{s}$-column
degrees satisfying $\theta\in O\left(\xi\right)$. Then we can multiply
$\mathbf{A}$ and $\mathbf{B}$ with a cost of $O^{\sim}(n^{2}m^{\omega-2}s)\subset O^{\sim}(n^{\omega}s)$,
where $s=\xi/n$ is the average of the entries of $\vec{s}$. 
\end{thm}

\subsection{Kernel Bases}

Let $\mathbb{K}$ be a field, $\mathbf{F}\in\mathbb{K}\left[x\right]^{m\times n}$
a matrix of polynomials. The kernel of $\mathbf{F}\in\mathbb{K}\left[x\right]^{m\times n}$
is the $\mathbb{F}\left[x\right]$-module 
\[
\left\{ \mathbf{p}\in\mathbb{K}\left[x\right]^{n}~|~\mathbf{F}\mathbf{p}=0\right\} 
\]
with a kernel basis of $\mathbf{F}$ being a basis of this module.
Kernel bases are closely related to order bases, as can be seen from
the following definitions. 
\begin{defn}
\label{def:kernelBasis}Given $\mathbf{F}\in\mathbb{K}\left[x\right]^{m\times n}$,
a polynomial matrix $\mathbf{N}\in\mathbb{K}\left[x\right]^{n\times*}$
is a (right) kernel basis of $\mathbf{F}$ if the following properties
hold: 
\begin{enumerate}
\item $\mathbf{N}$ is full-rank. 
\item $\mathbf{N}$ satisfies $\mathbf{F}\cdot\mathbf{N}=0$. 
\item Any $\mathbf{q}\in\mathbb{K}\left[x\right]^{n}$ satisfying $\mathbf{F}\mathbf{q}=0$
can be expressed as a linear combination of the columns of $\mathbf{N}$,
that is, there exists some polynomial vector $\mathbf{p}$ such that
$\mathbf{q}=\mathbf{N}\mathbf{p}$. 
\end{enumerate}
\end{defn}
Any pair of kernel bases $\mathbf{N}$ and $\mathbf{M}$ of $\mathbf{F}$
are column bases of each other and are unimodularly equivalent.

A $\vec{s}$-minimal kernel basis of $\mathbf{F}$ is just a kernel
basis that is $\vec{s}$-column reduced. 
\begin{defn}
Given $\mathbf{F}\in\mathbb{K}\left[x\right]^{m\times n}$, a polynomial
matrix $\mathbf{N}\in\mathbb{K}\left[x\right]^{n\times*}$ is a $\vec{s}$-minimal
(right) kernel basis of $\mathbf{F}$ if\textbf{ $\mathbf{N}$} is
a kernel basis of $\mathbf{F}$ and $\mathbf{N}$ is $\vec{s}$-column
reduced. We also call a $\vec{s}$-minimal (right) kernel basis of
$\mathbf{F}$ a $\left(\mathbf{F},\vec{s}\right)$-kernel basis.

We will need to use the following bound on the sizes of kernel bases
from \citep{ZLS2012}.\end{defn}
\begin{thm}
\label{thm:boundOfSumOfShiftedDegreesOfKernelBasis}Suppose $\mathbf{F}\in\mathbb{K}\left[x\right]^{m\times n}$
and $\vec{s}\in\mathbb{Z}_{\ge0}^{n}$ is a shift with entries bounding
the corresponding column degrees of $\mathbf{F}$. Then the sum of
the $\vec{s}$-column degrees of any $\vec{s}$-minimal kernel basis
of $\mathbf{F}$ is bounded by $\sum\vec{s}$. 
\end{thm}
We will also need the following result from \citep{ZLS2012} to compute
kernel bases by rows. 
\begin{thm}
\label{thm:continueComputingKernelBasisByRows}Let $\mathbf{G}=\left[\mathbf{G}_{1}^{T},\mathbf{G}_{2}^{T}\right]^{T}\in\mathbb{K}\left[x\right]^{m\times n}$
and $\vec{t}\in\mathbb{Z}^{n}$ a shift vector. If $\mathbf{N}_{1}$
is a $\left(\mathbf{G}_{1},\vec{t}\right)$-kernel basis with $\vec{t}$-column
degrees $\vec{u}$, and $\mathbf{N}_{2}$ is a $\left(\mathbf{G}_{2}\mathbf{N}_{1},\vec{u}\right)$-kernel
basis with $\vec{u}$-column degrees $\vec{v}$, then $\mathbf{N}_{1}\mathbf{N}_{2}$
is a $\left(\mathbf{G},\vec{t}\right)$-kernel basis $\vec{t}$-column
degrees $\vec{v}$. 
\end{thm}
Also recall the cost of kernel basis computation from \citep{ZLS2012}. 
\begin{thm}
\label{thm:costGeneral} Given an input matrix $\mathbf{F}\in\mathbb{K}\left[x\right]^{m\times n}$.
Let $\vec{s}=\cdeg\mathbf{F}$ and $s=\sum\vec{s}/n$ be the average
column degree of $\mathbf{F}$. Then a $\left(\mathbf{F},\vec{s}\right)$-kernel
basis can be computed with a cost of $O^{\sim}\left(n^{\omega}s\right)$
field operations.
\end{thm}

\section{The Algorithm}


\subsection{The Reduction}

The inverse algorithm from \citet{jeannerod-villard:05} computes
a matrix $\mathbf{A}$ that transforms $\mathbf{F}$ to a diagonal
matrix $\mathbf{B}=\mathbf{F}\mathbf{A}$. The inverse of $\mathbf{F}$
is then given by $\mathbf{A}\mathbf{B}^{-1}$.

To compute such a matrix $\mathbf{A}$, first consider a matrix $\mathbf{A}_{1}$
that transforms $\mathbf{F}$ to a matrix $\mathbf{R}=\mathbf{F}\mathbf{A}_{1}$
with just two diagonal blocks. More specifically, If we separate $\mathbf{F}$
to 
\[
\mathbf{F}=\left[\begin{array}{c}
\mathbf{F}_{u}\\
\hline \mathbf{F}_{d}
\end{array}\right],
\]
 with $\mathbf{F}_{u}$ and $\mathbf{F}_{d}$ consist of the upper
$\lceil n/2\rceil$ and lower $\left\lfloor n/2\right\rfloor $ rows
of $\mathbf{F}$, then a matrix $\mathbf{A}_{1}=\left[\mathbf{N}_{d},\mathbf{N}_{u}\right]$
consists of kernel bases $\mathbf{N}_{u}$ and $\mathbf{N}_{d}$ of
$\mathbf{F}_{u}$ and $\mathbf{F}_{d}$ gives
\begin{equation}
\mathbf{F}\cdot\mathbf{A}_{1}=\left[\begin{array}{c}
\mathbf{F}_{u}\\
\hline \mathbf{F}_{d}
\end{array}\right]\cdot\left[\mathbf{N}_{d},\mathbf{N}_{u}\right]=\left[\begin{array}{cc}
\mathbf{F}_{u}\mathbf{N}_{d} & \mathbf{F}_{u}\mathbf{N}_{u}\\
\mathbf{F}_{d}\mathbf{N}_{d} & \mathbf{F}_{d}\mathbf{N}_{u}
\end{array}\right]=\left[\begin{array}{cc}
\mathbf{R}_{u} & 0\\
0 & \mathbf{R}_{d}
\end{array}\right]~.\label{recurs}
\end{equation}
 If $\mathbf{F}$ is nonsingular, then the column dimensions of $\mathbf{N}_{d}$
and $\mathbf{N}_{u}$ match the row dimensions of $\mathbf{F}_{u}$
and $\mathbf{F}_{d}$, making the diagonal blocks $\mathbf{R}_{u}$
and $\mathbf{R}_{d}$ square. 

The same process can then be repeated recursively on $\mathbf{R}_{u}$
and $\mathbf{R}_{d}$, until we reach the base case where the dimension
becomes $1$. This gives a recursive algorithm shown in \prettyref{alg:matrixInverse},
which returns a sequence of matrices $\mathbf{A}_{1},\dots,\mathbf{A}_{\lceil\log n\rceil}$
that transforms the input matrix $\mathbf{F}$ to a diagonal matrix
$\mathbf{B}$. 

\begin{comment}
Note that rank sensitive computation for order basis is not as natural,
since for order basis we work with power series in general, whose
rank may never be truly computed. In addition, the computed basis
does not correspond to the rank. For these reasons, we do not pursue
rank sensitive computations for order basis. 
\end{comment}
\begin{algorithm}[t]
\caption{$\inv(\mathbf{F},\vec{s})$}
\label{alg:matrixInverse}

\begin{algorithmic}[1]
\REQUIRE{$\mathbf{F}\in\mathbb{K}\left[x\right]^{n\times n}$; $\vec{s}$ is
initially set to the column degrees of $\mathbf{F}$. It keeps track
of the degrees.}

\ENSURE{$\mathcal{A}=\left[\mathbf{A}_{1},\dots,\mathbf{A}_{\left\lceil \log n\right\rceil }\right],\mathbf{B}$
with $\mathbf{A}_{1},\dots,\mathbf{A}_{\left\lceil \log n\right\rceil },\mathbf{B}\in\mathbb{K}\left[x\right]^{n\times n}$
such that $\mathbf{A}_{1}\dots\mathbf{A}_{\left\lceil \log n\right\rceil }\mathbf{B}^{-1}=\mathbf{F}^{-1}$
if $\mathbf{F}$ is nonsingular, or fail if $\mathbf{F}$ is singular.}

\begin{comment}
\STATE{$\vec{s}:=$column degrees of $\mathbf{F}$;}
\end{comment}


\STATE{$\left[\mathbf{F}_{1}^{T},\mathbf{F}_{2}^{T}\right]:=\mathbf{F}^{T}$
with $\mathbf{F}_{1}$ consists of the top $\left\lceil n/2\right\rceil $
rows of $\mathbf{F}$;}

\STATE{\textbf{if }$\mathbf{F}=0$ \textbf{then} fail \textbf{endif};}

\begin{comment}
\IF{$\mathbf{F}=0$ }

\ifbody{fail; //$\mathbf{F}$ is singular}
\end{comment}


\STATE{\textbf{if }$n=1$ \textbf{then} \textbf{return} $\left\{ 1,\mathbf{F}\right\} $;
\textbf{endif};}

\begin{comment}
\IF{$n=1$ }

\ifbody{\RETURN $1,\mathbf{F}$;}
\end{comment}


\STATE{$\mathbf{N}_{1}:=\mnb(\mathbf{F}_{1},\vec{s})$;$\mathbf{N}_{2}:=\mnb(\mathbf{F}_{2},\vec{s})$;}\label{line:nullspaceBasisComputation}

\begin{comment}
\IF{$\columnDimension(\mathbf{N}_{1})\ne\left\lfloor n\right\rfloor $
\OR{} $\columnDimension(\mathbf{N}_{2})\ne\left\lceil n\right\rceil $ }

\ifbody{ fail; //$\mathbf{F}$ is singular}
\end{comment}


\STATE{\textbf{if }$\columnDimension(\mathbf{N}_{1})\ne\left\lfloor n/2\right\rfloor $
\OR{} $\columnDimension(\mathbf{N}_{2})\ne\left\lceil n/2\right\rceil $
\textbf{then} fail; \textbf{endif};}

\STATE{$\mathbf{R}_{1}:=\mathbf{F}_{1}\mathbf{N}_{2}$;$\mathbf{R}_{2}:=\mathbf{F}_{2}\mathbf{N}_{1}$;}\label{line:multiplyFN}

\STATE{$\left\{ \mathcal{A}^{(1)},\mathbf{H}_{1}\right\} :=\inv(\mathbf{R}_{1},\cdeg_{\vec{s}}\mathbf{N}_{2})$;
$\left\{ \mathcal{A}^{(2)},\mathbf{H}_{2}\right\} :=\inv(\mathbf{R}_{2},\cdeg_{\vec{s}}\mathbf{N}_{1})$;}

\STATE{$\mathcal{A}:=\left[\left[\mathbf{N}_{2},\mathbf{N}_{1}\right],\diag(\mathcal{A}_{1}^{(1)},\mathcal{A}_{1}^{(2)}),\dots,\diag(\mathcal{A}_{\left\lfloor \log n\right\rfloor }^{(1)},\mathcal{A}_{\left\lfloor \log n\right\rfloor }^{(2)})\right]$}

\STATE{\textbf{return} $\left\{ \mathcal{A},\diag\left([\mathbf{H}_{1},\mathbf{H}_{2}]\right)\right\} $;}
\end{algorithmic}
\end{algorithm}




\section{Complexity\label{sec:complexity}}


\subsection{\label{sub:sizes}Sizes of the intermediate results}

For \prettyref{alg:matrixInverse} to work efficiently, let us first
make sure that the sizes of matrices used in the algorithm are not
too big.

If the input matrix $\mathbf{F}$ has column degrees bounded by $\vec{s}$,
then we know from \prettyref{thm:boundOfSumOfShiftedDegreesOfKernelBasis}
that the sum of the $\vec{s}$-column degrees of the kernel bases
$\mathbf{N}_{1}$ is also bounded by $\sum\vec{s}$, that is, we have
$\sum\cdeg_{\vec{s}}\mathbf{N}_{1}\le\sum\vec{s}$. Similarly, we
also have $\sum\cdeg_{\vec{s}}\mathbf{N}_{2}\le\sum\vec{s}$.

For the sizes of the residual matrices $\mathbf{R}_{1}$ and $\mathbf{R}_{2}$,
we can apply \prettyref{lem:predictableDegree} to determine their
column degrees using the $\vec{s}$-column degrees of $\mathbf{N}_{1}$
and $\mathbf{N}_{2}$. That is, we have $\cdeg\mathbf{R}_{1}=\cdeg_{\vec{s}}\mathbf{N}_{1}$,
and $\cdeg\mathbf{R}_{2}=\cdeg_{\vec{s}}\mathbf{N}_{2}$. As a result,
we also have $\cdeg\mathbf{R}_{1}\le\sum\vec{s}$ and $\cdeg\mathbf{R}_{2}\le\sum\vec{s}$.
The residual matrices $\mathbf{R}_{1}$ and $\mathbf{R}_{2}$ are
then used as the input matrices of the new subproblems. 

Applying this analysis recursively, we see that for any subproblem,
if $\vec{t}$ is the list of column degrees of the input matrix, then
$\sum\vec{t}\le\sum\vec{s}$. The sum of the $\vec{t}$-column degrees
of each kernel basis computed is also bounded by $\sum\vec{t}\le\sum\vec{s}$.
The number of subproblems however doubles while the dimension halves
each time after each level of recursion.


\subsection{Computational cost of the algorithm}

The dominate costs of \prettyref{alg:matrixInverse} come from the
kernel basis computation and the matrix multiplications. These are
handled by the following lemmas.
\begin{lem}
The kernel basis computation at \prettyref{line:nullspaceBasisComputation}
costs $O^{\sim}(n^{\omega}s)$.\end{lem}
\begin{proof}
This follows from the kernel basis computation algorithm for polynomial
matrix with a shift recently reported by the authors \citet{ZLS2012}.
Here the shift is set to the column degrees of the input matrix.\end{proof}
\begin{lem}
The multiplications $\mathbf{R}_{1}:=\mathbf{F}_{1}\mathbf{N}_{2}$
and $\mathbf{R}_{2}:=\mathbf{F}_{2}\mathbf{N}_{1}$at \prettyref{line:multiplyFN}
cost $O^{\sim}(n^{\omega}s)$.\end{lem}
\begin{proof}
\prettyref{thm:boundOfSumOfShiftedDegreesOfKernelBasis} implies that
the sum of the $\vec{s}$-column degrees of $\mathbf{N}_{1}$ and
that of $\mathbf{N}_{2}$ are both bounded by $\xi$. The result then
follows directly from \prettyref{thm:multiplyUnbalancedMatrices}.
\end{proof}
We are now ready to look at the cost of \prettyref{alg:matrixInverse}.
\begin{thm}
\label{thm:inverseCost}\prettyref{alg:matrixInverse} costs $O^{\sim}\left(n^{\omega}s\right)$
field operations to compute an inverse of a nonsingular matrix $\mathbf{F}\in\mathbb{K}\left[x\right]^{n\times n}$.\end{thm}
\begin{proof}
Let the cost of the algorithm be $g(n)$. Then we have the following
recurrence relation:
\begin{eqnarray*}
g(n) & \in & O^{\sim}(n^{\omega}s)+g(\left\lceil n/2\right\rceil )+g(\left\lfloor n/2\right\rfloor )\\
 & \in & O^{\sim}(n^{\omega}s)+2g(\left\lceil n/2\right\rceil )\\
 & \in & O^{\sim}(n^{\omega}s).
\end{eqnarray*}
 Note that always rounding up $n/2$ to $\left\lceil n/2\right\rceil $
is no worse than assuming $n$ is a power of $2$. In other words,
the entries in the sequence $\left[\left\lceil n/2\right\rceil ,\left\lceil n/4\right\rceil ,\dots,1\right]$
is no larger than the corresponding entries in the sequence $\left[m/2,m/4,\dots,1\right]$,
where $m=2^{\left\lceil \log_{2}n\right\rceil }$ is the smallest
power of $2$ that is no less than $n$. 
\end{proof}
It is interesting to note that the algorithm costs just $O^{\sim}\left(n^{\omega}s\right)$
field operations, lower than $\Theta\left(n^{3}s\right)$, the number
of field elements generally required to represent the inverse matrix.
This is because the algorithm output a series of matrices $\left[\mathbf{A}_{1},\dots,\mathbf{A}_{\left\lceil \log n\right\rceil }\right]$
and $\mathbf{B}$ that require less space to store than their product.
Computing the product of the matrices $\left[\mathbf{A}_{1},\dots,\mathbf{A}_{\left\lceil \log n\right\rceil }\right]$
therefore cannot be lower than $O\left(n^{3}s\right)$. The following
section shows that a cost of $O^{\sim}(n^{3}s)$ can be achieved.


\subsection{Cost of multiplying output matrices}

To compute the product $\mathbf{A}=\mathbf{A}_{1}\dots\mathbf{A}_{\left\lceil \log n\right\rceil },$
we simply multiply the matrices in sequential order. Therefore, at
the $i$th step for $i$ from $1$ to $\left\lceil \log n\right\rceil -1$,
we multiply the product $\mathbf{M}_{i}=\mathbf{A}_{1}\mathbf{A}_{2}\cdots\mathbf{A}_{i}$
with $\mathbf{A}_{i+1}$. To determine the cost of this multiplication,
it is useful to first look at $\mathbf{A}_{i}$ and to determine its
size.

The matrix $\mathbf{A}_{i}$ consists of $2^{i-1}$ diagonal blocks
for $i\le\log n$, that is, $\mathbf{A}_{i}=\diag\left(\left[\mathbf{A}_{i}^{\left(1\right)},\mathbf{A}_{i}^{\left(2\right)},\dots,\mathbf{A}_{i}^{\left(2^{i-1}\right)}\right]\right)$,
each diagonal block $\mathbf{A}_{i}^{\left(k\right)}$ consists of
two kernel bases, $\mathbf{A}_{i,L}^{\left(k\right)}$ and $\mathbf{A}_{i,R}^{\left(k\right)}$,
computed in one of the subproblems. The product $\mathbf{M}_{i}=\mathbf{A}_{1}\mathbf{A}_{2}\cdots\mathbf{A}_{i}$
can be separated to $\mathbf{M}_{i}=\left[\mathbf{M}_{i}^{(1)},\dots,\mathbf{M}_{i}^{\left(2^{i}\right)}\right]$
with $2^{i}$ column blocks, where the column dimensions of $\mathbf{M}_{i}^{(2k)}$
and $\mathbf{M}_{i}^{(2k+1)}$ corresponds to the column dimensions
of $\mathbf{A}_{i,L}^{\left(k\right)}$ and $\mathbf{A}_{i,R}^{\left(k\right)}$
for each $k$. It is interesting to note that each $\mathbf{M}_{i}^{\left(k\right)}$
is in the kernel of a subset of the rows of the original input matrix
$\mathbf{F}$, which can be seen from the fact that the product $\mathbf{F}\mathbf{M}_{i}$
is a matrix with only diagonal blocks nonzero. In fact, $\mathbf{M}_{i}^{(k)}$
is also a $\vec{s}$-minimal kernel basis of a matrix consists of
a subset of the rows of the original input matrix $\mathbf{F}$. 
\begin{lem}
\label{lem:kernelBasesFromInverse}Let $\mathbf{F}_{i}^{(k)}$ be
a matrix consists of the rows of $\mathbf{F}$ that have the same
indices as the column indices of $\mathbf{M}_{i}^{\left(k\right)}$
in $\mathbf{M}_{i}$, and $\bar{\mathbf{F}}_{i}^{(k)}$ consists of
the remaining rows of $\mathbf{F}$. Then $\mathbf{M}_{i}^{\left(k\right)}$
is a $\vec{s}$-minimal kernel basis of $\bar{\mathbf{F}}_{i}^{(k)}$.\end{lem}
\begin{proof}
This can be seen by following the computation process, and apply \prettyref{thm:continueComputingKernelBasisByRows}.
One can see that the only rows not used in computing $\mathbf{M}_{i}^{\left(k\right)}$
are the rows from $\mathbf{F}_{i}^{(k)}$.
\end{proof}
Knowing that $\mathbf{M}_{i}^{\left(k\right)}$ is a $\vec{s}$-minimal
kernel basis of $\bar{\mathbf{F}}_{i}^{(k)}$ allows us to easily
obtain a bound on its size, which then allow us to obtain a bound
on the sizes of $\mathbf{A}_{i}$.
\begin{lem}
\label{lem:newKernelBasisSize}If $\vec{t}=\cdeg_{\vec{s}}\mathbf{M}_{i}^{\left(k\right)}$,
then $\sum\vec{t}\le\sum\vec{s}$. In addition, we have $\cdeg_{\vec{t}}\mathbf{A}_{i+1,L}^{\left(k\right)}\le\vec{t}$
and $\cdeg_{\vec{t}}\mathbf{A}_{i+1,R}^{\left(k\right)}\le\vec{t}$. \end{lem}
\begin{proof}
The first statement follows directly from \prettyref{thm:boundOfSumOfShiftedDegreesOfKernelBasis},
since $\mathbf{M}_{i}^{\left(k\right)}$ is a $\vec{s}$-minimal kernel
basis of $\bar{\mathbf{F}}_{i}^{(k)}$ and $\cdeg\bar{\mathbf{F}}_{i}^{(k)}\le\cdeg\mathbf{F}=\vec{s}$.

Now let us look at the second statement. From $\cdeg\mathbf{F}_{i}^{(k)}\le\vec{s}$
and $\cdeg_{\vec{s}}\mathbf{M}_{i}^{\left(k\right)}=\vec{t}$, \prettyref{lem:predictableDegree}
implies that the matrix $\mathbf{R}=\mathbf{F}_{i}^{(k)}\cdot\mathbf{M}_{i}^{\left(k\right)}$
satisfies $\cdeg\mathbf{R}\le\vec{t}$. The matrix $\mathbf{R}$ is
then used as the input matrix for computing two kernel bases in $\mathbf{A}_{i+1}^{\left(k\right)}=\left[\mathbf{A}_{i+1,L}^{\left(k\right)},\mathbf{A}_{i+1,R}^{\left(k\right)}\right]$.
By \prettyref{thm:boundOfSumOfShiftedDegreesOfKernelBasis} these
two kernel bases satisfy $\cdeg_{\vec{t}}\mathbf{A}_{i+1,L}^{\left(k\right)}\le\vec{t}$
and $\cdeg_{\vec{t}}\mathbf{A}_{i+1,R}^{\left(k\right)}\le\vec{t}$. \end{proof}
\begin{lem}
The multiplications $\mathbf{A}=\mathbf{A}_{1}\dots\mathbf{A}_{\left\lceil \log n\right\rceil }$
can be done with a cost of $O^{\sim}(n^{3}s)$ .\end{lem}
\begin{proof}
First, let us determine the cost of the multiplication $\mathbf{M}_{i}^{\left(k\right)}\mathbf{A}_{i+1,L}^{\left(k\right)}$.
The matrix $\mathbf{M}_{i}^{\left(k\right)}$ has dimension $n\times O(n/2^{i})$.
The matrix $\mathbf{A}_{i+1,L}^{\left(k\right)}$ has dimension $O(n/2^{i})\times O(n/2^{i})$.
(Big $O$ notation is used here because $n/2^{i}$ may not be an integer.)
Since $\sum\cdeg\mathbf{M}_{i}^{\left(k\right)}\le\sum\cdeg_{\vec{s}}\mathbf{M}_{i}^{\left(k\right)}=\sum\vec{t}\le\sum\vec{s}=ns$
and $\sum\cdeg_{\vec{t}}\mathbf{A}_{i+1,L}^{\left(k\right)}\le\sum\vec{t}\le ns$,
we can apply \prettyref{thm:multiplyUnbalancedMatrices} to obtain
a cost of $O^{\sim}\left(2^{i}\left(n/2^{i}\right)^{\omega}\left(ns\right)/\left(n/2^{i}\right)\right)=O^{\sim}\left(2^{i\left(2-\omega\right)}n^{\omega}s\right).$
The multiplication $\mathbf{M}_{i}^{\left(k\right)}\mathbf{A}_{i+1,R}^{\left(k\right)}$
also has the same cost. Doing such multiplications for $k$ from 1
to $2^{i}$ then gives the product $\mathbf{M}_{i}\mathbf{A}_{i+1}$.
Therefore, the cost of multiplication $\mathbf{M}_{i}\mathbf{A}_{i+1}$
is $O^{\sim}\left(2^{i\left(3-\omega\right)}n^{\omega}s\right)$.
Take $\omega=3$, we get a cost of $O^{\sim}\left(n^{3}s\right)$.
Doing this $\left\lceil \log n\right\rceil -1$ times for $i$ from
1 to $\left\lceil \log n\right\rceil -1$ to compute $\mathbf{A}=\mathbf{M}_{\left\lceil \log n\right\rceil }=\mathbf{A}_{1}\dots\mathbf{A}_{\left\lceil \log n\right\rceil }$
still has an overall cost of $O^{\sim}\left(n^{3}s\right)$.
\end{proof}

\section{Largest Invariant Factors}

\label{sec:invariants}

The largest invariant factor of a matrix of polynomials ${\bf F}$
is the minimal degree monic polynomial $p$ having the property that
$p\cdot{\bf F}^{-1}$ is a matrix polynomial. Alternative it is defined
as the ratio of the determinant and the gcd of all $n-1$ minors of
$\mathbf{F}$ and coincides with the last diagonal entry of the Smith
normal form of ${\bf F}$. For generic matrices this is the same as
the determinant while when $\mathbf{F}=xI-V$ for a scalar matrix
$V$ then the largest invariant factor is the same as the minimal
polynomial of $V$. In this section we will show how the largest invariant
factor can be determined from the matrix ${\bf B}$ of Algorithm \ref{alg:matrixInverse}.
 We use \emph{column basis} as a tool, so let us first briefly look
at column basis. 

A column basis of $\mathbf{F}$ is a basis for the $\mathbb{K}\left[x\right]$-module
\[
\left\{ \mathbf{F}\mathbf{p}~|~\mathbf{p}\in\mathbb{K}\left[x\right]^{n}~\right\} ~.
\]
Such a basis can be represented as a full rank matrix $\mathbf{T}\in\mathbb{K}\left[x\right]^{m\times r}$
whose columns are the basis elements. The following lemma from \citep{ZL2013,zhou:phd2012}
allows us to factor any matrix as the product of a column basis and
a kernel basis.
\begin{lem}
\label{lem:columnBasis} Let $\mathbf{F}\in\mathbb{K}\left[x\right]^{m\times n}$
with rank $r$. Suppose $\mathbf{N}\in\mathbb{K}\left[x\right]^{n\times(n-r)}$
is a right kernel basis of $\mathbf{F}$ and $\mathbf{G}\in\mathbb{K}\left[x\right]^{r\times n}$
is a left kernel basis of $\mathbf{N}$. Then $\mathbf{F}=\mathbf{T}\cdot\mathbf{G}$
with $\mathbf{T}\in\mathbb{K}\left[x\right]^{m\times r}$ a column
basis of $\mathbf{F}$. 
\end{lem}
The following result from Chapter $8$ of \citep{zhou:phd2012} shows
when a matrix can be made unimodular by adding more rows.
\begin{lem}
\label{lem:unimodularCompletionCondition}A unimodular completion
of $\mathbf{F}$ exists if and only if $\mathbf{F}$ has unimodular
column bases. 
\end{lem}
The following lemma gives a relationship between the determinant of
a column basis and the gcd of the minors of a matrix.
\begin{lem}
\label{lem:determinantOfColumnBasisAndGcdOfMinors}If $\mathbf{G}\in\mathbb{K}\left[x\right]^{\left(n-1\right)\times n}$
has full row rank and $\mathbf{C}$ is a column basis of $\mathbf{G}$,
then the gcd of all $\left(n-1\right)\times\left(n-1\right)$ minors
of $\mathbf{G}$ equals $c\det\mathbf{C}$ for some $c\in\mathbb{K}$.\end{lem}
\begin{proof}
Using \prettyref{lem:columnBasis} we can factor $\mathbf{G}$ as
$\mathbf{G}=\mathbf{C}\mathbf{H}$, where $\mathbf{H}$ has an identity
matrix gcd and can be unimodularly completed by \prettyref{lem:unimodularCompletionCondition}.
Let $\mathbf{h}=\left[h_{1},\dots,h_{n}\right]$ be a row vector that
gives the unimodular matrix 
\[
\bar{\mathbf{H}}=\begin{bmatrix}\mathbf{H}\\
\mathbf{h}
\end{bmatrix},
\]
 then $\det\bar{\mathbf{H}}=\sum\left(h_{i}\det\mathbf{H}_{i}\right)$,
where $\det\mathbf{H}_{i}$ is the $\left(n-1\right)\times\left(n-1\right)$
minor in $\mathbf{H}$ corresponding to $h_{i}$. Since $\bar{\mathbf{H}}$
is unimodular, we have $\det\bar{\mathbf{H}}=\sum\left(h_{i}\det\mathbf{H}_{i}\right)\in\mathbb{K}\backslash0$.
We can in fact scale $h$ to make $\det\bar{\mathbf{H}}=\sum\left(h_{i}\det\mathbf{H}_{i}\right)=1$.
This means that the minors $\det\mathbf{H}_{i}$ are coprime. Now
putting back the column basis $\mathbf{C}$ gives the result.\end{proof}
\begin{thm}
Suppose that ${\mathbf{B}}=\mbox{ diag }(b_{1},\ldots,b_{n})$ denotes
the final diagonal matrix polynomial in Algorithm \ref{alg:matrixInverse}.
Then 
\[
\mbox{ Largest Invariant Factor of }{\mathbf{F}}=\mbox{lcm}(b_{1},\ldots,b_{n}).
\]
\end{thm}
\begin{proof}
Let $\mathbf{a}_{k}$ be the $k$th column of $\mathbf{A}$, and $\bar{\mathbf{F}}^{\left(k\right)}$
be the matrix $\mathbf{F}$ with the $k$th row removed. Then from
\prettyref{lem:kernelBasesFromInverse} we know that $\mathbf{a}^{\left(k\right)}$
is a $\vec{s}$-minimal kernel basis of $\bar{\mathbf{F}}^{(k)}$.
Let $\mathbf{f}_{k}$ be the $k$th row of $\mathbf{F}$. Then the
$k$th diagonal entry of the output matrix $\mathbf{B}$ is $b_{k}=\mathbf{f}_{k}\mathbf{a}_{k}$.

We can apply Lemma 9.1 from \citep{zhou:phd2012} to unimodularly
transform $\mathbf{F}$ to 
\[
P\mathbf{F}\mathbf{U}=\begin{bmatrix}\mathbf{C}_{k}\\
 & b_{k}
\end{bmatrix},
\]
where $P$ is a permutation matrix that switches the $k$th row of
$\mathbf{F}$ to the last row (for simplicity), and \textbf{$\mathbf{C}_{k}$
}is a column basis of $\bar{\mathbf{F}}^{\left(k\right)}$. This gives
$\det\mathbf{F}=b_{k}\det\mathbf{C}_{k}$ or $b_{k}=\det\mathbf{F}/\det\mathbf{C}_{k}$.

From \prettyref{lem:determinantOfColumnBasisAndGcdOfMinors} $\det\mathbf{C}_{k}$
gives the gcd of all $\left(n-1\right)\times\left(n-1\right)$ minors
of $\bar{\mathbf{F}}^{\left(k\right)}$. Hence $\gcd\left(\det\mathbf{C}_{1},\dots,\det\mathbf{C}_{n}\right)$
equals the gcd of all $\left(n-1\right)\times\left(n-1\right)$ minors
of $\mathbf{F}^{\left(k\right)}$. Therefore, $\det\mathbf{F}/\gcd\left(\det\mathbf{C}_{1},\dots,\det\mathbf{C}_{n}\right)$
gives the largest invariant factor of the smith normal form of $\mathbf{F}$.
The result then follows from 
\[
\frac{\det\mathbf{F}}{\gcd\left(\det\mathbf{C}_{1},\dots,\det\mathbf{C}_{n}\right)}=\lcm\left(\frac{\det\mathbf{F}}{\det\mathbf{C}_{1}},\frac{\det\mathbf{F}}{\det\mathbf{C}_{2}},\dots,\frac{\det\mathbf{F}}{\det\mathbf{C}_{n}}\right)=\lcm\left(b_{1},\dots,b_{n}\right).
\]

\end{proof}



\section{Conclusion}

In this paper we have extended the algorithm of Jeannerod and Villard
for the inversion of a matrix polynomial. The algorithm is deterministic
and has essentially optimal complexity for inversion. In addition,
we show how the algorithm can also be used to find the largest invariant
factor of a matrix of polynomials.

Again, it is interesting to note that \prettyref{alg:matrixInverse}
costs only $O^{\sim}\left(n^{\omega}s\right)$ and represents the
inverse with $O\left(n^{2}s\log n\right)$ space. It is possible that
this representation is useful in some applications. For example, if
we wish to multiply another low degree matrix or a row vector $\mathbf{H}$
by $\mathbf{F}^{-1}$, representing $\mathbf{F}^{-1}=\mathbf{A}\mathbf{B}^{-1}$
requires us to multiply $\mathbf{H}$ with a high degree matrix $\mathbf{A}$.
This can be more expensive than the multiplication using the representation
$\mathbf{F}^{-1}=\mathbf{A}_{1}\mathbf{A}_{2}\cdots\mathbf{A}_{\left\lceil \log n\right\rceil }\mathbf{B}^{-1}$,
then $\mathbf{H}\mathbf{F}^{-1}=\mathbf{H}\mathbf{A}_{1}\mathbf{A}_{2}\cdots\mathbf{A}_{\left\lceil \log n\right\rceil }\mathbf{B}^{-1}$,
which is less expensive. It may be interesting to look for other applications
where this smaller representation is useful.

\bibliographystyle{plainnat}
\bibliography{paper}

\end{document}
